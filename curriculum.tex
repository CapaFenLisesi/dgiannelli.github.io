\documentclass[11pt,a4paper]{moderncv}

% moderncv themes
%\moderncvtheme[blue]{casual}                 % optional argument are 'blue' (default), 'orange', 'red', 'green', 'grey' and 'roman' (for roman fonts, instead of sans serif fonts)
\moderncvtheme[grey]{classic}                % idem

%\usepackage[T1]{fontenc}
% character encoding
\usepackage[utf8x]{inputenc}                   % replace by the encoding you are using
\usepackage[english]{babel}
\usepackage{xcolor}
\usepackage{xpatch}
\xpatchcmd\cventry{.}{}{}{}

% adjust the page margins
\usepackage[scale=0.8]{geometry}
\recomputelengths                             % required when changes are made to page layout lengths

\fancyfoot{} % clear all footer fields
\fancyfoot[LE,RO]{\thepage}           % page number in "outer" position of footer line
\fancyfoot[RE,LO]{\footnotesize } % other info in "inner" position of footer line

% personal data
\firstname{Giovanni}
\familyname{Iannelli}
\title{Curriculum Vitae}               % optional, remove the line if not wanted
\address{via Galvani 2}{56126 Pisa}{Italy}    % optional, remove the line if not wanted
\mobile{+39 347 4279310}                    % optional, remove the line if not wanted
\email{iannelli.giovanni@gmail.com}                      % optional, remove the line if not wanted
%\extrainfo{additional information (optional)} % optional, remove the line if not wanted
\photo[80pt][0pt]{picture}                         % '64pt' is the height the picture must be resized to and 'picture' is the name of the picture file; optional, remove the line if not wanted

%\nopagenumbers{}                             % uncomment to suppress automatic page numbering for CVs longer than one page

\begin{document}
\maketitle

\section{Personal informations}
\cvline{Birth}{\small 12 August 1992, Bolzano (Bozen), Italy\normalsize}
\cvline{Citizenship}{\small Italian\normalsize}
\cvline{Status}{\small Graduate student\normalsize}

\section{Education}
\cventry{2015-Present}{Master student in Theoretical Physics}{Universit\`a di Pisa}{Italy}{}{\small {During my first year of Master's studies in theoretical physics, I'm focusing on quantum field theory, statistical physics and their computational aspects. In particular, I'm attending a course whose syllabus covers many algorithms used in computational theoretical physics, such as simulated Ising model and path integral computing, both in quantum mechanics and quantum field theory}}

\cventry{2011-2015}{Bachelor's Degree in Physics}{Universit\`a di Pisa}{Italy}{}{} % arguments 3 to 6 are optional

\cvline{Grade}{103/110}
\cvline{Thesis}{\small{\emph{Bosons in a harmonic trap: Bose-Einstein condensation and a Path Integral Monte Carlo algorithm}}}
\cvline{Supervisor}{\small Ettore Vicari, professor at Universit\`a di Pisa}
\cvline{Description}{\small I discussed the Bose-Einstein condensation in a system of non interacting bosons affected by a harmonic potential. I showed that the partition function of the system could be expressed in terms of Feynman path integrals. Considering a path integral in statistical mechanics as a weighted sum of paths, I presented a method to sample the paths corrisponding to the positions (or momenta) of bosons}

\cventry{2006-2011}{High School Diploma}{Liceo Scientifico Leonardo da Vinci}{Milan}{Italy}{}
\cvline{Grade}{84/100}

\section{Programming experience} 
\cvline{C/C++}{\small I studied C language at university, as it is part of the program of my Bachelor's Degree. Then, I studied C++ to write object-oriented programs and make usage of ROOT (CERN) libraries}
\cvline{FORTRAN~90}{\small I've used FORTRAN 90 language and OpenMP and MPI on top of it to write parallel PDE solvers, linear algebra operations and Monte Carlo simulations}
\cvline{Python}{\small I've used Python for scripting, Python Scientific packages (Numpy, SciPy) to test algorithms or to perform quick-to-write computations, Matplotlib for 2D plotting
and Mayavi for 3D plotting}
\cvline{Git}{\small I've a good knowledge of Git VCS, and, throught my university experience, I've got experience in collaborating with other people on the same project}
\cvline{OS}{\small I've got experience in Linux based operative systems, especially Arch Linux based and Debian based, and I've confidence with Bash shell commands}
\cvline{Others}{\small I've got experience in text editing with VIM, typesetting with \LaTeX, auto-building with Makefile, symbolic computing with Wolfram Mathematica, and plotting with Gnuplot}

\section{Languages}
\hspace{25mm}\small Self-assessment European level \textcolor{blue}{\href{http://europass.cedefop.europa.eu/en/resources/european-language-levels-cefr}{CEFR}} (C2 maximum evaluation)\normalsize
\vspace{5mm}

\begin{tabular}{p{67mm} p{40mm} p{40mm} p{20mm}}
& \textbf{Understanding} & \textbf{Speaking} & \textbf{Writing} \\ \end{tabular}

\begin{tabular}{p{67mm} p{20mm} p{20mm} p{20mm} p{20mm} p{20mm}}
& \small Listening & \small Reading & \small Interaction & \small Production & \\ \end{tabular}

\vspace{3mm}
%lvl should be in this range A1 < A2 < B1 < B2 < C1 < C2 
\cvlanguage{Italian}{Mother tongue}{
	\begin{tabular}{p{20mm} p{20mm} p{20mm} p{20mm} p{21mm}}
		C2 & C2 & C2 & C2 & C2
	\end{tabular}}
\cvlanguage{English}{Advanced}{
	\begin{tabular}{p{20mm} p{20mm} p{20mm} p{20mm} p{21mm}}
		B2 & C1 & B2 & B2 & C1
	\end{tabular}}

\section{Other courses and achievements}
\cventry{2015}{High Performance Scientific Computing}{Professor: Randall J. LeVeque, University of Washington}{Coursera \href{https://en.wikipedia.org/wiki/Massive_open_online_course}{MOOC}}{\textcolor{blue}{\href{https://www.coursera.org/course/scicomp}{Link}} to course page}
{\small After an introduction to Unix like shell, Git, Python, FORTRAN 90, Makefile, OpenMP and MPI FORTRAN API, this online course aims to teach how to write parallelized programs to solve typical scientific computing problems, such as Linear Algebra optimizations, PDE solving and Monte Carlo simuations. During this course, there were homeworks and a project assignment, but there wasn't a final exam}

\cventry{2015}{Statistical Mechanics, Algorithms and Computation}{Professor: Werner Krauth, \'Ecole Normale Sup\'erieure}{Coursera \href{https://en.wikipedia.org/wiki/Massive_open_online_course}{MOOC}}{\textcolor{blue}{\href{https://www.coursera.org/learn/statistical-mechanics}{Link}} to course page}
{\small This online course is an introduction to Monte Carlo algorithms and their applications in statistical mechanics and quantum statistical mechanics using the path integral and density matrices framework. At the end of this course, an exam was scheduled, and I completed it as it is stated in this coursera.org \textcolor{blue}{\href{http://dgiannelli.github.io/coursera_smac.pdf}{Certificate}}}

\cventry{2010}{First Certificate in English}{University of Cambridge}{}{}{I took this exam when I was in high school, and it certificates the Council of Europe Level B2 in English}
\end{document}
